\documentclass[conference]{IEEEtran} % default

\usepackage{cite} % default

\usepackage{amsmath,amssymb,amsfonts} % default

\usepackage{hyperref} % for hyperlinks obviously
\usepackage{cleveref}
% The following formats are used so that when I call \cref{label_1} or \cref{label_1,label_2} or \crefrange{label_1}{label_4}: 
    % 1. I never get eq. or eqs. before my bracketed numbers.
    % 2. The numbers are bracketed.

% For single equation references
\crefformat{equation}{(#2#1#3)}
\Crefformat{equation}{(#2#1#3)}
% For multiple equation references
\crefmultiformat{equation}
{(#2#1#3)} % First reference
{ and (#2#1#3)} % Middle references
{ and (#2#1#3)} % Last reference
{ and (#2#1#3)} % Last reference
% For a range of equation references (e.g., (1)-(3))
\crefrangeformat{equation}{(#3#1#4) to (#5#2#6)}

\usepackage{algorithmic} % default
\usepackage{graphicx} % default

\usepackage[subpreambles=true]{standalone}
\usepackage{import}
\usepackage{subfiles} % for invoking figures from within a section vs by running the main file

% \usepackage{hyperref} % for hyperlinks obviously

\usepackage{subcaption} % for caption of figures with multiple plots

% \usepackage{lipsum} 
\usepackage{textcomp} % default
\usepackage{xcolor} % default

\usepackage{ulem} % for strikethrough-ing text

\title{A Spatially Distributed Multi-Period Optimal Power Flow Study with Distributed Battery Units}

% Redefine the IEEEauthorrefmark command for using numbers as superscripts
\makeatletter
\newcommand{\mysup}[1]{\@fnsymbol{#1}}
\makeatother

% \author{Nathan Gray,~\IEEEmembership{Student Member~IEEE,}
%         Subho Paul,~\IEEEmembership{Member~IEEE,}
%         Anamika Dubey,~\IEEEmembership{Senior Member~IEEE,} \\
%         Anjan Bose,~\IEEEmembership{Fellow~IEEE,}
%         Md. Touhiduzzaman,~\IEEEmembership{Member~IEEE,}
% 	  and James Ogle,~\IEEEmembership{Member~IEEE} 
%     \thanks{The authors acknowledge support from the U.S. Department of Energy under Contract DE-AC05-76RL01830
    
%  N. Gray, A. Dubey, and A. Bose are with the  School of Electrical Engineering and Computer Science, Washington State University, Pullman, WA, USA. S. Paul is with the Department of Electrical Engineering, Indian Institute of Technology (BHU), Varanasi, UP, India. Md. Touhiduzzaman and J. Ogle are with the Energy and Environment Directorate, Pacific Northwest National Laboratory (PNNL), Richland, WA, USA. Email: (nathan.gray, anamika.dubey, bose)@wsu.edu, subho.eee@iitbhu.ac.in, (md.touhiduzzaman, james.ogle)@pnnl.gov
%  }
%  }


\author{
    \IEEEauthorblockN{
        Aryan Ritwajeet Jha\mysup{1}, \textit{SIEEE},
        Subho Paul\mysup{2}, \textit{MIEEE},
        Anamika Dubey\mysup{1}, \textit{SMIEEE}
        }
\IEEEauthorblockA{\IEEEauthorrefmark{1}\textit{School of Electrical Engineering \& Computer Science},
\textit{Washington State University},
Pullman, WA, USA\\
\IEEEauthorrefmark{2}\textit{Department of Electrical Engineering},
\textit{Indian Institute of Technology (BHU) Varanasi},
Varanasi, UP, India\\
\IEEEauthorrefmark{1}\{aryan.jha, anamika.dubey\}@wsu.edu, 
\IEEEauthorrefmark{2}\{subho.eee\}@iitbhu.ac.in}

\thanks{%This work is supported  by the U.S. Department offfffff
%Energy’s Office of Energy Efficiency and Renewable Energy (EERE) under the Solar Energy Technologies Office Award Number DE-EE-0008774 (awarded to Sukumar Kamalasadan).
 Authors ack}\vspace{-7mm}}

% \thanks{This work is supported  by the U.S. Department of
% Energy’s Office of Energy Efficiency and Renewable Energy (EERE) under the Solar Energy Technologies Office Award Number DE-EE-0008774 (awarded to Sukumar Kamalasadan).
% S. Paul is with the  School of Electrical Engineering and Computer Science, Washington State University, Pullman, Washington 99163, USA (e-mail: subho.paul@wsu.edu). K. Murari is with the Electrical Engineering and Computer Science Department, University of Toledo, Toledo, Ohio 43606, USA (e-mail: krishna.murari@utoledo.edu).
% }

\begin{document}

\maketitle


\begin{abstract}

The growing presence of battery-associated distributed energy resources (DERs) in distribution networks necessitates the development of multi-period optimal power flow (MPOPF) strategies. Generally, the MPOPF frameworks are developed as mixed integer non-convex programming (MINCP) and solved centrally. However, the main limitation of centralized MPOPF (MPCOPF) is its longer solution time, a typical solution time is in the order of \(10^3\) to \(10^4\) seconds. This article proposes a spatially distributed MPOPF (MPDOPF) to overcome such deficiencies. Initially, the OPF problem is developed as a single-phase MPCOPF for a distribution network consisting of distributed DERs and battery units. Later the original large-scale centralized OPF problem is split into multiple sub-problems, which are solved in parallel by sharing boundary voltage and power data with the neighboring agents by following the directives of the Equivalent Network Approximation method (ENApp). The performance characterization of the proposed MPDOPF framework is conducted using the IEEE 123 bus test system. This analysis offers insights into the superiority of distributed MPOPF frameworks over centralized ones concerning solution time.

\end{abstract}

\begin{IEEEkeywords}
Batteries, distribution network, distributed energy resources, equivalent network approximation (ENApp) 
\end{IEEEkeywords}

\import{../sections/intro/}{intro.tex}

\subfile{../sections/theory/theory.tex}
% \import{../sections/theory/}{theory.tex}

\import{../sections/simulation/}{simulation.tex}

% \import{../sections/results/}{results.tex}
\subfile{../sections/results/results.tex}
% \import{../sections/conclusion/}{conclusion.tex}

\subfile{../sections/conclusion/conclusion.tex}

% \cite{bfm01,Nazir2018Jun,Nazir2019Jun,ddp_sugar_01,Qian2014Jul}


\bibliographystyle{IEEEtran}
\bibliography{bibFile}

\end{document}
