\documentclass[conference]{IEEEtran} % default

\usepackage{cite} % default
\usepackage{amsmath,amssymb,amsfonts} % default
\usepackage{algorithmic} % default
\usepackage{graphicx} % default

\usepackage[subpreambles=true]{standalone}
\usepackage{import}

% \usepackage{lipsum} 

\usepackage{textcomp} % default
\usepackage{xcolor} % default

\title{A Fast Multi-Period Optimal Power Flow Algorithm based on Spatially Distributed Computing}

% Redefine the IEEEauthorrefmark command for using numbers as superscripts
\makeatletter
\newcommand{\mysup}[1]{\@fnsymbol{#1}}
\makeatother

\author{
    \IEEEauthorblockN{
        Aryan Ritwajeet Jha\mysup{1}, \textit{Student Member, IEEE},
        Subho Paul\mysup{1}, \textit{Member, IEEE},
        Anamika Dubey\mysup{1}, \textit{Senior Member, IEEE}
        }
\IEEEauthorblockA{\IEEEauthorrefmark{1}\textit{School of Electrical Engineering \& Computer Science} \\
\textit{Washington State University}\\
Pullman, WA\\
\{aryan.r.jha, subho.paul, anamika.dubey\}@wsu.edu}
}

\begin{document}

\maketitle


\begin{abstract}
This document is a model and instructions for \LaTeX.
This and the IEEEtran.cls file define the components of your paper [title, text, heads, etc.]. *CRITICAL: Do Not Use Symbols, Special Characters, Footnotes, 
or Math in Paper Title or Abstract.
\end{abstract}

\begin{IEEEkeywords}
component, formatting, style, styling, insert
\end{IEEEkeywords}

\import{../sections/intro/}{intro.tex}

\import{../sections/theory/}{theory.tex}

\import{../sections/simulation/}{simulation.tex}

\import{../sections/results/}{results.tex}

\import{../sections/conclusion/}{conclusion.tex}

\cite{bfm01,Nazir2018Jun,Nazir2019Jun,ddp_sugar_01,Qian2014Jul}


\bibliographystyle{IEEEtran}
\bibliography{bibFile}

\end{document}
