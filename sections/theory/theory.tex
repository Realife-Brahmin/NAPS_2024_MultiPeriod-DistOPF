\documentclass{article}


\usepackage{cite} % default

\usepackage{amsmath,amssymb,amsfonts} % default
\usepackage{cleveref}

\usepackage{algorithmic} % default
\usepackage{graphicx} % default

\usepackage[subpreambles=true]{standalone}
\usepackage{import}

% \usepackage{lipsum} 
\usepackage{textcomp} % default
\usepackage{xcolor} % default
% Any packages or configurations specific to this section
\usepackage{lipsum}

\definecolor{darkgreen}{rgb}{0.0,0.0,0.0}

\begin{document}

\section{Theory}

\subsection{Notations}

\subsection{Centralized Multi-Period OPF with Batteries}

On similar lines to the branch flow equations in \cite{bfm01}, the network is modeled as a function of time, considering the interaction of batteries.

% \begin{gather}
	\begin{align}
		{p_j^t} & = {\sum_{(j, k) \in \mathcal{L}} P_{jk}^t - \sum_{(i, j) \in \mathcal{L}}\left\{P_{ij}^t - r_{ij}l_{ij}^t\right\} - P_{d_j}^t + P_{c_j}^t} && \label{eq_Pij} \\ 
		{q_j^t} & = {\sum_{(j, k) \in \mathcal{L}} Q_{jk}^t - \sum_{(i, j) \in \mathcal{L}}\left\{Q_{ij}^t - x_{ij}l_{ij}^t\right\} - q_{D_j}^t - q_{B_j}^t} && \label{eq_Qij} \\ 
		{v_j^t} & = {v_{i}^t +  \left\{r_{ij}^2 + x_{ij}^2\right\}l_{ij}^t - 2(r_{ij}P_{ij}^t + x_{ij}Q_{ij}^t)} && \\
		{l_{ij}^t} & = {\frac{(P_{ij}^{t})^2 + (Q_{ij}^{t})^2}{v_i^t}}
		% {\color{darkgreen}{ B_{j}^{t} }} &\color{darkgreen}{=} { \color{darkgreen}{ B_{j}^{t-1} + \Delta t  \eta_c P_{c_j}^t - \Delta t\frac{1}{\eta_d} P_{d_j}^t } } \\
		% { B_{j}^{0} } &= { 0.5(soc_{max}+soc_{min})E_{Rated} = 0.625E_{Rated}} \\
		% % { B_{j}^{T} } &= { B_{j}^{0}} \\
		% {where,} & {} \\
		% {(i, j)} &: {\text{Branch connecting nodes $i$ and $j$}} \\
		% {p_j^t} &= {p_D{_j}^t - p_L{_j}^t} \\
		% {q_j^t} &= {-q_L{_j}^t} \\
		% {t} &= {\{1, 2, \ldots T\}}
	\end{align}
% \end{gather}

where $P^{t}_{ij}, Q^{t}_{ij}, l^{t}_{ij}$ denote the sending-end real power, reactive power and the square of the magnitude of the current flowing in the branch $(i, j)$ respectively. $v^{t}_{j}$ denotes the square of the magnitude of the voltage at node $j$. The superscript $t$ specifies the time-period for the corresponding variable. It may be noted that for a radial distribution system, each node $j$ can have only one `parent' node $i$, and thus the summation for the second term in equations \Cref{eq_Pij, eq_Qij} may be dropped.

\subsection*{(Integer Constraint Relaxed) Naive Brute Force Full Optimization Model - Full Horizon}



\begin{gather}
	\begin{align}
		\min_{
		\substack{
		P_{ij}^t, Q_{ij}^t, v_{j}^t, l_{ij}^t, \\
		q_{D_j}^t, B_{j}^{t}, P_{c_j}^t, P_d{_j}^t, q_{B_j}^t}
		} 
		\vspace{-15ex} % Adjust the space here
		& \quad
		\sum_{t = 1}^{T} \sum_{(i, j) \in \mathcal{L}} (r_{ij}l_{ij}^t) \\
		&+ \alpha \sum_{t = 1}^{T} \sum_{j \in \mathcal{B}} \left\{ (1- \eta_c)P_{c_j}^t + \left(\frac{1}{\eta_d}-1\right) P_{d_j}^t \right\} \\
		&+ \gamma \sum_{j \in \mathcal{B}} \left\{ \left(B^{T}_{j} - B_{ref_{j}}]\right)^2\right\} \\
	% \end{align}
	% \begin{align}
		\text{s.t.} & {}\nonumber \\
		{p_j^t} & = {\sum_{(j, k) \in \mathcal{L}} P_{jk}^t - \sum_{(i, j) \in \mathcal{L}}\left\{P_{ij}^t - r_{ij}l_{ij}^t\right\} - P_{d_j}^t + P_{c_j}^t} && \\
		{q_j^t} & = {\sum_{(j, k) \in \mathcal{L}} Q_{jk}^t - \sum_{(i, j) \in \mathcal{L}}\left\{Q_{ij}^t - x_{ij}l_{ij}^t\right\} - q_{D_j}^t - q_{B_j}^t} && \\
		{v_j^t} & = {v_{i}^t +  \left\{r_{ij}^2 + x_{ij}^2\right\}l_{ij}^t - 2(r_{ij}P_{ij}^t + x_{ij}Q_{ij}^t)} && \\
		{l_{ij}^t} & = {\frac{(P_{ij}^{t})^2 + (Q_{ij}^{t})^2}{v_i^t}} \\
		{\color{darkgreen}{ B_{j}^{t} }} &\color{darkgreen}{=} { \color{darkgreen}{ B_{j}^{t-1} + \Delta t  \eta_c P_{c_j}^t - \Delta t\frac{1}{\eta_d} P_{d_j}^t } } \\
		{ B_{j}^{0} } &= { 0.5(soc_{max}+soc_{min})E_{Rated} = 0.625E_{Rated}} \\
		% { B_{j}^{T} } &= { B_{j}^{0}} \\
		{where,} & {} \\
		{(i, j)} &: {\text{Branch connecting nodes $i$ and $j$}} \\
		{p_j^t} &= {p_D{_j}^t - p_L{_j}^t} \\
		{q_j^t} &= {-q_L{_j}^t} \\
		{t} &= {\{1, 2, \ldots T\}}
	\end{align}
\end{gather}

\subsection{ENApp based Distributed Multi-Period OPF with Batteries}

\end{document}
