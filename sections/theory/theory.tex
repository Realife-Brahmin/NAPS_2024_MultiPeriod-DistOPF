\documentclass{article}


\usepackage{cite} % default

\usepackage{amsmath,amssymb,amsfonts} % default
\usepackage{cleveref}
\crefformat{equation}{(#2#1#3)} % to not put eqn. when I use \cref{<single reference>}
\crefrangeformat{equation}{(#3#1#4) to (#5#2#6)} % to not put eqns. when I use \cref{<multiple references>}

\usepackage{algorithmic} % default
\usepackage{graphicx} % default

% \usepackage[subpreambles=true]{standalone}
\usepackage{import}

% \usepackage{lipsum} 
\usepackage{textcomp} % default
\usepackage{xcolor} % default
% Any packages or configurations specific to this section
\usepackage{lipsum}

\begin{document}

\section{Problem Formulation}

\subsection{Notations}
In this study, the distribution network is accounted as a tree (connected graph) having \(N\) number of buses (indexed with \(i\), \(j\), and \(k\)) and the study is conducted for \(T\) time steps (indexed by \(t\)). The distribution line connecting two buses \(i\) and \(j\) are denoted by {\(ij\)} (having resistance and reactance of \(r_{ij}\) ohm and \(x_{ij}\) ohm, respectively) and magnitude of the current flowing through the line at time \(t\) is denoted by \(I_{ij}^t\) (\(l_{ij}^t=\left(I_{ij}^t\right)^2\)). The voltage magnitude of bus \(i\) at time \(t\) is given by \(V_i^t \in [V_{min},V_{max}]\) (\(v_i^t=\left(V_i^t\right)^2\)). Apparent power demand at a node \(j\) at time \(t\) is \(s_L{_j}^t\) (\(=p_L{_j}^t+\textit{j}q_L{_j}^t\)). The uncontrolled active power generation from the DER present at bus \(j\) at time step \(t\) is denoted by \(p_D{_j}^t\) and controlled reactive power dispatch from the DER inverter is \(q_D{_j}^t\). Static capacitance attached to a node $j$ is denoted by $q_{C_{j}}$. The apparent power flow through line {\(ij\)} at time step \(t\) is \(S_{ij}^t\) (\(=P_{ij}^t+\textit{j}Q_{ij}^t\)). The battery energy level is \(B_j^t\). Charging and discharging active power from battery inverter (of capacity \(S_j^t\)) are denoted by \(P_{c_j}^t\) and \(P_{d_j}^t\), respectively. The reactive power support of the battery inverter is \(q_{B_j}^t\). Rated apparent powers of DERs and Batteries at node $j$ are denoted by $S_{D_{Rated, j}}$ and $S_{B_{Rated, j}}$ respectively.

\subsection{Centralized Multi-Period OPF with Batteries}
The OPF problem aims to minimize the total network loss for the entire time period, as written below,
\begin{equation}
    \min {\sum_{t = 1}^{T} \sum_{(i, j) \in \mathcal{L}} (r_{ij}l_{ij}^t)}
\end{equation}
Subject to the following constraints,
\begin{align}
    {p_j^t} & = {\sum_{(j, k) \in \mathcal{L}} P_{jk}^t - \left\{P_{ij}^t - r_{ij}l_{ij}^t\right\} - P_{d_j}^t + P_{c_j}^t} && \label{eq:Pij} \\ 
    {q_j^t} & = {\sum_{(j, k) \in \mathcal{L}} Q_{jk}^t - \left\{Q_{ij}^t - x_{ij}l_{ij}^t\right\} - q_{D_j}^t - q_{B_j}^t} && \label{eq:Qij} \\ 
    {p_j^t} &= {p_D{_j}^t - p_L{_j}^t} \label{eq:pj}\\
    {q_j^t} &= {q_C{_j} -q_L{_j}^t} \label{eq:qj}\\
    {v_j^t} & = {v_{i}^t +  \left\{r_{ij}^2 + x_{ij}^2\right\}l_{ij}^t - 2(r_{ij}P_{ij}^t + x_{ij}Q_{ij}^t)} \label{eq:vj} && \\
    {l_{ij}^t} & = {\frac{(P_{ij}^{t})^2 + (Q_{ij}^{t})^2}{v_i^t}} \label{eq:lij} && \\
    { B_{j}^{t} } &= {  B_{j}^{t-1} + \Delta t  \eta_c P_{c_j}^t - \Delta t\frac{1}{\eta_d} P_{d_j}^t } && \\
    { v^{t}_{j} } &\in { \left[ V^{2}_{Min}, V^{2}_{Max} \right]} \label{eq:lim_vj} && \\
    { l^{t}_{ij} } &\in { \left[ 0, I^{2}_{ij, rated}
    \right] } \label{eq:lim_lij} && \\
    { B^{t}_{j} } &\in { \left[ 0.30E_{Max_{j}}, 0.95E_{Max_{j}} \right] } \label{eq:lim_Bj} && \\
    { P^{t}_{c_{j}}, P^{t}_{d_{j}} }
    &\in
    { \left[ 0, P_{B_{Max_{j}}} \right]} \label{eq:lim_PcPdj} && \\
    { q^{t}_{B_{j}} } 
    &\in 
    { \left[-\sqrt{ {S_{B_{Rated, j}}}^2 - {P_{B_{Max_{j}}}}^2}, \sqrt{ {S_{B_{Rated, j}}}^2 - {P_{B_{Max_{j}}}}^2}\right] } \label{eq:qBj} &&
\end{align}

The distribution network is represented with the help of the branch power flow equations \cref{eq:Pij,eq:Qij,eq:pj,eq:qj,eq:vj,eq:lij}. Constraints \cref{eq:Pij} and \cref{eq:Qij} signify the active and reactive power balance equations. The net active and reactive power injections at any bus \(j\) are represented by \cref{eq:pj} and \cref{eq:qj} respectively. The KVL equation is represented by \cref{eq:vj}, while the equation describing the relationship between current magnitude, voltage magnitude and apparent power magnitude is \cref{eq:lij}. \cref{eq:qDj} describes the two-quadrant controlled reactive power support of DERs, where   

Node $i$ denotes the `parent' node of node $j$, which itself may be the parent of a set of $k$ `children' nodes (the set may contain one, many or even zero nodes, if $j$ is a leaf node). It may be noted that for a radial distribution system, each node $j$ can have only one `parent' node $i$.

\subsection*{(Integer Constraint Relaxed) Naive Brute Force Full Optimization Model - Full Horizon}



\subsection{ENApp based Distributed Multi-Period OPF with Batteries}

\end{document}
