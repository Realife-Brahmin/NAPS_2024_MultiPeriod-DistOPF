\documentclass{article}


\usepackage{cite} % default

\usepackage{amsmath,amssymb,amsfonts} % default
\usepackage{cleveref}

\usepackage{algorithmic} % default
\usepackage{graphicx} % default

% \usepackage[subpreambles=true]{standalone}
\usepackage{import}

% \usepackage{lipsum} 
\usepackage{textcomp} % default
\usepackage{xcolor} % default
% Any packages or configurations specific to this section
\usepackage{lipsum}

\definecolor{darkgreen}{rgb}{0.0,0.0,0.0}

\begin{document}

\section{Theory}
\textcolor{red}{FIRST LINE OF THEORY}
\textcolor{green}{ANOTHER LINE OF THEORY}

\subsection{Notations}
In this study, the distribution network is accounted as a tree (connected graph) having \(N\) number of buses (indexed with \(i\), \(j\), and \(k\)) and the study is conducted for \(T\) time steps (indexed by \(t\)). The distribution line connecting two buses \(i\) and \(j\) are denoted by {\(ij\)} and magnitude of the current flowing through the line at time \(t\) is denoted by \(I_{ij}^t\) (\(l_{ij}^t=\left(I_{ij}^t\right)^2\)). The voltage magnitude of bus \(i\) at time \(t\) is given by \(V_i^t\) (\(v_i^t=\left(V_i^t\right)^2\)). 

\subsection{Centralized Multi-Period OPF with Batteries}

On similar lines to the branch flow equations in \cite{bfm01}, the network is modeled as a function of time, considering the interaction of batteries.

\begin{align}
    {p_j^t} & = {\sum_{(j, k) \in \mathcal{L}} P_{jk}^t - \sum_{(i, j) \in \mathcal{L}}\left\{P_{ij}^t - r_{ij}l_{ij}^t\right\} - P_{d_j}^t + P_{c_j}^t} && \label{eq:Pij} \\ 
    {q_j^t} & = {\sum_{(j, k) \in \mathcal{L}} Q_{jk}^t - \sum_{(i, j) \in \mathcal{L}}\left\{Q_{ij}^t - x_{ij}l_{ij}^t\right\} - q_{D_j}^t - q_{B_j}^t} && \label{eq:Qij} \\ 
    {v_j^t} & = {v_{i}^t +  \left\{r_{ij}^2 + x_{ij}^2\right\}l_{ij}^t - 2(r_{ij}P_{ij}^t + x_{ij}Q_{ij}^t)} \label{eq:vj} && \\
    {l_{ij}^t} & = {\frac{(P_{ij}^{t})^2 + (Q_{ij}^{t})^2}{v_i^t}} \label{eq:lij}
\end{align}

where $P^{t}_{ij}, Q^{t}_{ij}, l^{t}_{ij}$ denote the sending-end real power, reactive power and the square of the magnitude of the current flowing in the branch $(i, j)$ respectively. $v^{t}_{j}$ denotes the square of the magnitude of the voltage at node $j$. The superscript $t$ specifies the time-period for the corresponding variable. Node $i$ denotes the `parent' node of node $j$, which itself may be the parent of a set of $k$ `children' nodes (the set may contain one, many or even zero nodes, if $j$ is a leaf node). It may be noted that for a radial distribution system, each node $j$ can have only one `parent' node $i$, and thus the summation for the second term in equations \Cref{eq:Pij,eq:Qij,eq:vj,eq:lij} may be dropped.

\subsection*{(Integer Constraint Relaxed) Naive Brute Force Full Optimization Model - Full Horizon}

\begin{align}
    \min_{
    % \substack{
    % P_{ij}^t, Q_{ij}^t, v_{j}^t, l_{ij}^t, \\
    % q_{D_j}^t, B_{j}^{t}, P_{c_j}^t, P_d{_j}^t, q_{B_j}^t}
    } 
    \vspace{-15ex} % Adjust the space here
    & \quad
    {\sum_{t = 1}^{T} \sum_{(i, j) \in \mathcal{L}} (r_{ij}l_{ij}^t)}  \nonumber\\
    {} &+ {\alpha \sum_{t = 1}^{T} \sum_{j \in \mathcal{B}} \left\{ (1- \eta_c)P_{c_j}^t + \left(\frac{1}{\eta_d}-1\right) P_{d_j}^t \right\}} \nonumber\\
    {} &+ {\gamma \sum_{j \in \mathcal{B}} \left\{ \left(B^{T}_{j} - B_{ref_{j}}]\right)^2\right\}} \\
    \text{s.t.} & {}\nonumber \\
    {} & {\cref{eq:Pij,eq:Qij,eq:vj,eq:lij}} \\
    {\color{darkgreen}{ B_{j}^{t} }} &\color{darkgreen}{=} { \color{darkgreen}{ B_{j}^{t-1} + \Delta t  \eta_c P_{c_j}^t - \Delta t\frac{1}{\eta_d} P_{d_j}^t } } \\
    % { B_{j}^{0} } &= { 0.5(soc_{max}+soc_{min})E_{Rated} = 0.625E_{Rated}} \\
    {where,} & {} \\
    {(i, j)} &: {\text{Branch connecting nodes $i$ and $j$}} \\
    {p_j^t} &= {p_D{_j}^t - p_L{_j}^t} \\
    {q_j^t} &= {-q_L{_j}^t} \\
    {t} &= {\{1, 2, \ldots T\}}
\end{align}

\subsection{ENApp based Distributed Multi-Period OPF with Batteries}

\end{document}
