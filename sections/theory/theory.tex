% \documentclass{article}
\documentclass[../../outputs/main.tex]{subfiles}

\usepackage{cite} % default

\usepackage{amsmath,amssymb,amsfonts} % default
\usepackage{cleveref}
% The following formats are used so that when I call \cref{label_1} or \cref{label_1,label_2} or \crefrange{label_1}{label_4}: 
    % 1. I never get eq. or eqs. before my bracketed numbers.
    % 2. The numbers are bracketed.

% For single equation references
\crefformat{equation}{(#2#1#3)}
\Crefformat{equation}{(#2#1#3)}
% For multiple equation references
\crefmultiformat{equation}
{(#2#1#3)} % First reference
{ and (#2#1#3)} % Middle references
{ and (#2#1#3)} % Last reference
{ and (#2#1#3)} % Last reference
% For a range of equation references (e.g., (1)-(3))
\crefrangeformat{equation}{(#3#1#4) to (#5#2#6)}


\usepackage{algorithmic} % default
\usepackage{graphicx} % default

% \usepackage[subpreambles=true]{standalone}
\usepackage{import}

% \usepackage{lipsum} 
\usepackage{textcomp} % default
\usepackage{xcolor} % default
% Any packages or configurations specific to this section
\usepackage{lipsum}

\begin{document}

\section{Problem Formulation}

\subsection{Notations}
In this study, the distribution network is accounted as a tree (connected graph) having \(N\) number of buses (indexed with \(i\), \(j\), and \(k\)) and the study is conducted for \(T\) time steps (indexed by \(t\)), each of interval length $\Delta t$. The distribution line connecting two buses \(i\) and \(j\) are denoted by {\(ij\)} (having resistance and reactance of \(r_{ij}\) ohm and \(x_{ij}\) ohm, respectively) and magnitude of the current flowing through the line at time \(t\) is denoted by \(I_{ij}^t\) (\(l_{ij}^t=\left(I_{ij}^t\right)^2\)). The voltage magnitude of bus \(i\) at time \(t\) is given by \(V_i^t \in [V_{min},V_{max}]\) (\(v_i^t=\left(V_i^t\right)^2\)). Apparent power demand at a node \(j\) at time \(t\) is \(s_L{_j}^t\) (\(=p_L{_j}^t+\textit{j}q_L{_j}^t\)). The uncontrolled active power generation from the DER present at bus \(j\) at time step \(t\) is denoted by \(p_D{_j}^t\) and controlled reactive power dispatch from the DER inverter is \(q_D{_j}^t\). Static capacitance attached to a node $j$ is denoted by $q_{C_{j}}$. The apparent power flow through line {\(ij\)} at time step \(t\) is \(S_{ij}^t\) (\(=P_{ij}^t+\textit{j}Q_{ij}^t\)). The battery state of charge (soc) or energy level is \(B_j^t\). Charging and discharging active power from battery inverter (of apparent power capacity \(S^{t}_{R, j}\) ) are denoted by \(P_{c_j}^t\) and \(P_{d_j}^t\), respectively. The total state of charge capacity of the batteries are denoted by $E_{R, j}$, and the Rated battery powers are denoted by $P_{B_{R, j}}$. The reactive power support of the battery inverter is \(q_{B_j}^t\). Rated apparent powers of DERs and Batteries at node $j$ are denoted by $S_{D_{R, j}}$ and $S_{B_{R, j}}$ respectively.

\subsection{Centralized Multi-Period OPF with Batteries}
The OPF problem given in \cref{eq:genCost_withSCD} aims to minimize the cost of power borrowed from the substation for the entire horizon. The incorporation of an additional 'Battery Loss' term helps us bypass using binary (integer) constraints for modelling the operation of batteries, which make the optimization problem harder to solve. The term still ensures the complementarity of charging and discharging operations for any battery during a particular time period \cite{Nazir2018Jun, Nazir2019Jun, Nazir2021Sep}.

% \begin{equation}
%     \min {\sum_{t = 1}^{T} \left[ C^t P^t_{Subs} + \alpha \sum_{j \in \mathcal{B}} \left\{ (1-\eta_C)P^t_{C_j} + \left( \frac{1}{\eta_D} - 1 \right) P^t_{D_j}\right\} \right] } \label{eq:genCost_withSCD}
% \end{equation}
\begin{equation}
    \begin{split}
        \min \sum_{t = 1}^{T} \left[ C^t P^t_{Subs} + \alpha \sum_{j \in \mathcal{B}} \left\{ (1-\eta_C)P^t_{C_j} \right. \right. \\
        \left. \left. + \left( \frac{1}{\eta_D} - 1 \right) P^t_{D_j} \right\} \right]
    \end{split}
    \label{eq:genCost_withSCD}
\end{equation}


Subject to the constraints \crefrange{eq:RealPowerBalanceNodej}{eq:qDj} given below:
\begin{align}
    % {p_j^t} & = {\sum_{(j, k) \in \mathcal{L}} P_{jk}^t - \left\{P_{ij}^t - r_{ij}l_{ij}^t\right\} - P_{d_j}^t + P_{c_j}^t} && \label{eq:RealPowerBalanceNodej} \\ 
    {0} &= {\sum_{(j, k) \in \mathcal{L}} \left\{P_{jk}^t\right\} - \left(P_{ij}^t - r_{ij}l_{ij}^t\right) - \left(P_{d_j}^t - P_{c_j}^t\right) - p^t_{D_j} + p^t_{L_j}} && \label{eq:RealPowerBalanceNodej} \\ 
    % {q_j^t} & = {\sum_{(j, k) \in \mathcal{L}} Q_{jk}^t - \left\{Q_{ij}^t - x_{ij}l_{ij}^t\right\} - q_{D_j}^t - q_{B_j}^t} && \label{eq:Qij} \\ 
    {0} &= {\sum_{(j, k) \in \mathcal{L}} \left\{Q_{jk}^t\right\} - \left(Q_{ij}^t - x_{ij}l_{ij}^t\right) - q_{D_j}^t - q_{B_j}^t + q^t_{L_j}} && \label{eq:ReactivePowerBalanceNodej} \\ 
    % {p_j^t} &= {p_D{_j}^t - p_L{_j}^t} \label{eq:pj}\\
    % {q_j^t} &= {q_C{_j} -q_L{_j}^t} \label{eq:qj}\\
    % {v_j^t} & = {v_{i}^t +  \left\{r_{ij}^2 + x_{ij}^2\right\}l_{ij}^t - 2(r_{ij}P_{ij}^t + x_{ij}Q_{ij}^t)} \label{eq:vj} && \\
    {0} &= {v_{i}^t - v_j^t - 2(r_{ij}P_{ij}^t + x_{ij}Q_{ij}^t) + \left\{r_{ij}^2 + x_{ij}^2\right\}l_{ij}^t} \label{eq:KVL-branch-ij} && \\
    % {l_{ij}^t} & = {\frac{(P_{ij}^{t})^2 + (Q_{ij}^{t})^2}{v_i^t}} \label{eq:lij} && \\
    {0} &= {(P_{ij}^{t})^2 + (Q_{ij}^{t})^2 - l_{ij}^t v_i^t} \label{eq:ApparentPowerEquationBFM} && \\
    {P^t_{Subs}} &\geq {0} \label{eq:substationRealPowerLimits} \\
    { v^{t}_{j} } &\in { \left[ V^{2}_{min}, V^{2}_{max} \right]} \label{eq:lim_vj} && \\
    { l^{t}_{ij} } &\in { \left[ 0, I^{2}_{R, ij}
    \right] } \label{eq:lim_lij} && \\
    { B_{j}^{t} } &= {  B_{j}^{t-1} + \Delta t  \eta_c P_{c_j}^t - \Delta t\frac{1}{\eta_d} P_{d_j}^t } \label{eq:Bj} &&  \\
    { B^{t}_{j} } &\in { \left[ soc_{min}E_{R, j}, soc_{max}E_{R, j} \right] } \label{eq:lim_Bj} && \\
    { P^{t}_{c_{j}}, P^{t}_{d_{j}} }
    &\in
    { \left[ 0, P_{B_{R_{j}}} \right]} \label{eq:lim_PcPdj} && \\
    { q^{t}_{B_{j}} } 
    &\in 
    { \left[-\sqrt{ {S_{B_{R, j}}}^2 - {P_{B_{R, j}}}^2}, \sqrt{ {S_{B_{R, j}}}^2 - {P_{B_{R, j}}}^2}\right] } \label{eq:qBj} && \\
    { q^{t}_{D_{j}} } 
    &\in
    { \left[-\sqrt{ {S_{D_{R, j}}}^2 - {p^{t}_{D_{j}}}^2}, \sqrt{ {S_{D_{R, j}}}^2 - {p^{t}_{D_{j}}}^2}\right] } \label{eq:qDj} &&
\end{align}

\begin{table}[h!]
    \centering
    \caption{Parameter Values}
    \begin{tabular}{|l|c|}
    \hline
    \textbf{Parameter} & \textbf{Value} \\ \hline
    $V_{min}, V_{max}$ & 0.95, 1.05 \\ \hline
    $P_{DER_{Rated_j}}$ & $0.33 P_{L_{Rated_j}}$ \\ \hline
    $S_{DER_{Rated_j}}$ & $1.2 P_{DER_{Rated_j}}$ \\ \hline
    $P_{B_{Rated_j}}$ & $0.33 P_{L_{Rated_j}}$ \\ \hline
    $B_{Rated_j}$ & $T_{fullCharge} \times P_{B_{Rated_j}}$ \\ \hline
    $T_{fullCharge}$ & 4 h \\ \hline
    $\Delta t$ & 1 h \\ \hline
    $\eta_C, \eta_D$ & 0.95, 0.95 \\ \hline
    $soc_{min}, soc_{max}$ & 0.30, 0.95 \\ \hline
    $\alpha$ & 0.001 \\ \hline
    \end{tabular}
    \label{table:parameter-values}
\end{table}

The distribution network is represented with the help of the branch power flow equations \crefrange{eq:RealPowerBalanceNodej}{eq:ApparentPowerEquationBFM}. Constraints \cref{eq:RealPowerBalanceNodej,eq:ReactivePowerBalanceNodej} signify the active and reactive power balance at node $j$. 
% The net active and reactive power injections at any bus \(j\) are represented by \cref{eq:pj,eq:qj} respectively. 
The KVL equation for branch $(i, j)$ is represented by \cref{eq:KVL-branch-ij}, while the equation describing the relationship between current magnitude, voltage magnitude and apparent power magnitude for nodes $i$ and $j$ is \cref{eq:ApparentPowerEquationBFM}. The limits of node voltage and branch current are enforced via \cref{eq:lim_vj,eq:lim_lij}. The trajectory of the state of charge of batteries versus time is given by \cref{eq:Bj} and is the only class of constraints in this paper coupling the optimal power flow problem in time. Battery charging and discharging efficiency values used in this paper are $\eta_c = 95\%$ and $\eta_d = 95\%$ \textcolor{red}{literature?}. For a safe and sustainable operation of the batteries \textcolor{red}{based on what literature?}, the state of charge $B^{t}_{j}$ is constrained to be within some percentage limits of the rated battery soc capacity, as given in \cref{eq:lim_Bj}. In this paper, we're using $soc_{min} = 30\%$ and $soc_{max} = 95\%$. Similarly, battery charging and discharging powers should not exceed its rated power capacity, as given by \cref{eq:lim_PcPdj}.  \cref{eq:qDj,eq:qBj} describe the limits for two-quadrant operation of the controlled reactive power support of DERs and Batteries respectively. It may be noted that while both of these limits are non-controllable, only the limits for DERs are time-varying, due to $p^{t}_{D_j}$ component. For this simulation study, the limits for battery reactive support have been curtailed, i.e. the bounds of the limit have been artificially set smaller than what would be physically permissible. The reason for doing so was to avoid a non-linear inequality coupling decision variables. \textcolor{red}{Should I specify this justification?}

% Node $i$ denotes the `parent' node of node $j$, which itself may be the parent of a set of $k$ `children' nodes (the set may contain one, many or even zero nodes, if $j$ is a leaf node). It may be noted that for a radial distribution system, each node $j$ can have only one `parent' node $i$.

\subsection*{(Integer Constraint Relaxed) Naive Brute Force Full Optimization Model - Full Horizon}



\subsection{ENApp based Distributed Multi-Period OPF with Batteries}

\end{document}
