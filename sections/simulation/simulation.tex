% \documentclass{article}
\documentclass[../../outputs/main.tex]{subfiles}

% Any packages or configurations specific to this section
\usepackage{lipsum}
\usepackage{graphicx}

\begin{document}

\section{Case Study Demonstration}
% The energy price data is taken from the ComEd hourly live prices data \cite{comedLivePrices} for 10 November 2023. 

\subsection{Simulation Data: IEEE 123 Bus Test System}
The case studies are conducted on the balanced three-phase version of the IEEE 123 bus test system, which has $85$ Load Nodes. Additionally, $20 \% \, (17)$ and $30 \% \, (26)$ of these load nodes also contain reactive power controllable Solar photovoltaics (PVs) and batteries respectively. Their ratings are as per \Cref{table:parameter-values}. To demonstrate the effectiveness of the proposed algorithm, the test system has been divided into four areas like \cite{Sadnan}, as shown in \Cref{fig:ieee123-four-area-figure}. It is assumed that a horizon-wide forecast for loads $p^t_L$, solar power output $p^t_D$ and cost of substation power  $C^t$ is available to the distribution grid operator. Initially, the study is carried out for 5-time steps with input data shown in \Cref{fig:inputCurve-5}. The five-time step workflow is described below. 

\begin{table}[t]
    \centering
    \caption{Parameter Values}
    \begin{tabular}{|l|c|}
    \hline
    \textbf{Parameter} & \textbf{Value} \\ \hline
    $V_{min}, V_{max}$ & 0.95, 1.05 \\ \hline
    $p_{\ds D_{R_j}}$ & $0.33 p_{\ds L_{R_j}}$ \\ \hline
    $S_{D_{R_j}}$ & $1.2 p_{\ds D_{R_j}}$ \\ \hline
    $P_{B_{R_j}}$ & $0.33 p_{\ds L_{R_j}}$ \\ \hline
    $B_{R_j}$ & $T_{fullCharge} \times P_{B_{R_j}}$ \\ \hline
    $T_{fullCharge}$ & 4 h \\ \hline
    $\Delta t$ & 1 h \\ \hline
    $\eta_c, \eta_d$ & 0.95, 0.95 \\ \hline
    $soc_{min}, soc_{max}$ & 0.30, 0.95 \\ \hline
    $\alpha$ & 0.001 \\ \hline
    \end{tabular}
    \label{table:parameter-values}
\end{table}

\begin{figure}[t]
    \centering
    \includegraphics[width=\linewidth]{../figures/ieee123-FourAreas-pv20-batt30.png}
    \caption{IEEE 123 Node System Divided Into Four Areas}
    \label{fig:ieee123-four-area-figure}
\end{figure}

% To showcase the workflow of the proposed algorithm, simulations were run for a $5$ time-period horizon. 

\begin{figure}[t]
    \centering
    \includegraphics[height=0.25\textheight]{../figures/T5-inputCurves/InputCurves_Horizon_5.png}
    \caption{Forecasts for Demand Power, Irradiance and Cost of Substation Power over a 5 Hour Horizon}
    \label{fig:inputCurve-5}
\end{figure}

\def\ds{\rule{0pt}{1.5ex}} % this will lower the subscript by that amount, useful for $p_{D_{R_j}}$ where otherwise p and D appear to be almost at the same level.



\subsection{Simulation Workflow}

All simulations were set up in MATLAB 2023a including both the high level algorithms as well as calls to the optimization solver. MATLAB's \texttt{fmincon} function was used to parse the nonlinear nonconvex optimization problem described by \crefrange{eq:genCost_withSCD}{eq:modelEndsHere-and-lim_Bj} in tandem with the SQP optimization algorithm to solve it. From the completed simulations, the resultant optimal control variables were obtained, and were passed through an OpenDSS engine (already configured with system data and forecast values) in order to check for the feasiblity of the results. The associated code may be found \href{https://github.com/Realife-Brahmin/MultiPeriod-DistOPF-Benchmark}{here}.

% \href{https://t.ly/dUCfC}{here}. % shortened link

\end{document}
